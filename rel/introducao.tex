\newpage
\section{Introdução}
Este trabalho tem por objetivo a implementação das operações de rotacionamento, redirecionamento e aplicação de filtros de imagens. A filtragem aplicada a uma imagem digital é uma operação local que modifica os valores dos níveis digitais de cada pixel da imagem considerando o contexto atual do pixel.

O processo de filtragem é feito utilizando matrizes denominadas máscaras, as quais são aplicadas sobre a imagem. Pela filtragem, o valor de cada pixel da imagem é modificado utilizando-se uma operação de vizinhança, ou seja, uma operação que leva em conta os níveis digitais dos pixels vizinhos e o próprio valor digital do pixel considerado. Nas definições de vizinhança começamos por considerar que: um pixel qualquer da imagem digital é vizinho dele mesmo.

Os filtros espaciais podem ser classificados em passa-baixa, passa-alta ou passa-banda. Os dois primeiros são os mais utilizados em processamento de imagens. O filtro passa-banda é mais utilizado em processamentos específicos, principalmente para remover ruídos periódicos.

\section{Objetivos da filtragem}
Extração de ruídos da imagem;
Homogeneização da imagem ou de alvos específicos;
Melhora na discriminação de alvos da imagem;
Detecção de bordas entre alvos distintos presentes na imagem;
Detecção de formas.

\section{Conceitos básicos da filtragem digital}
No processo de filtragem digital utiliza-se uma operação de convolução de uma máscara pela imagem digital. Isso equivale a uma operação que "passeia" sobre toda imagem original modificando seus valores de acôrdo com os valores da imagem originais e os pesos da máscara.
A máscara utilizada é também uma imagem, em geral quadrada, menor que a imagem original. Os valores da imagem máscara são utilizados como pesos a serem aplicados sobre os níveis digitais dos pixels da imagem original.